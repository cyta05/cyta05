\documentclass[a4paper,12pt]{article}
\usepackage[utf8]{inputenc}
\usepackage{graphicx}
\usepackage{hyperref}
\usepackage{geometry}
\usepackage{authblk}

\geometry{left=2.5cm, right=2.5cm, top=3cm, bottom=3cm}

% Paquete para referencias en formato APA
\usepackage[style=apa,sorting=nyt]{biblatex}
\addbibresource{references.bib} % Archivo .bib con las referencias

\title{Ciencia Abierta con IA en la Web Semántica}
\author{
Perissé, Marcelo Claudio $^{1}$ \\
\small $^1$ Ciencia y Técnica Administrativa, Buenos Aires, Argentina \\
\small ORCID: \texttt{0000-0002-2766-5104} \\
\small \texttt{cyta@cyta.ar} 
\and
ChatGPT $^{2}$ \\
\small $^2$ OpenAI, California, EE. UU. \\
\small ORCID: \texttt{0000-0000-0000-0000} \\
\small \texttt{email2@ejemplo.com} 
\and
Autor Tres $^{3}$\\
\small $^3$ Afiliación 3, Ciudad, País \\
\small ORCID: \texttt{0000-0000-0000-0000} \\
\small \texttt{email3@ejemplo.com} 
}
\date{\today}


\begin{document}

\maketitle

\begin{abstract}
Este es el resumen del informe, donde se presenta una visión general del contenido y los principales hallazgos.

\textbf{Keywords:} keyword 1; keyword 2; keyword 3, keyword 4; keyword 5
\end{abstract}

\section{Introducción}
La visión que impulsa la investigación es la de estudiar cómo se construyen Cadenas Cooperativas de Valor → Cada recurso (OAI-PMH, IA, ChatGPT, OpenAI, RDF, SPARQL, JSON-LD, RSS) donde se interconecta y potencia la sistematicidad en la ciencia, mediante la organización sistemádica de los procesos de revisión de literatura científica, curación, exposición racional de las ideas elaboradas, correcto planteo de problemas o elaboración de hipótesis, y edición de los trabajos desarrollados.\cite{Sabino1992}

El proyecto busca principalmente documentar la integración de Ciencia Abierta, Inteligencia Artificial, Web Semántica, y Ciencia Ciudadana en el ecosistema de CyTA 5.0; consecuentemente se presenta como un estudio de caso donde observamos particularidades para poder generalizar.


Nuestro fin es el de desarrollar un modelo de colaboración IA-Humano, para establecer una Arquitectura de Razonamiento Compartido.

El objetivo es desarrollar un modelo de colaboración IA-Humano para establecer una Arquitectura de Razonamiento Compartido, aplicable a los procesos de enseñanza activa y aprendizaje basado en proyectos

Estrategia, implementar una nueva dinámica, a los procesos de publicar, revisar y curar, que permita pasar de gestionar información sobre repositorios de información, a un \textbf{ecosistema vivo de conocimiento en evolución}.


\section{Metodología}
Explicación de los métodos utilizados para llevar a cabo la investigación. Consecuentemente se realiza una descripción, lo suficientemente detallada y completa, de los procedimientos utilizados en la investigación, que les permita a los lectores evaluar la forma en que los resultados fueron obtenidos.
Si la investigación incluía aparatos, instrumentos o reactivos, se requerirá de una descripción de ellos, su diseño y la precisión de los instrumentos.

\section{Resultados}
Presentación de los hallazgos obtenidos en la investigación, basados en los métodos establecidos y que serán esenciales para fundamentar las conclusiones. 


\section{Discusión}
El autor interpreta los resultados más relevantes obtenidos que ya se han fundamentado en dicha sección; y por lo tanto no introduce ningún nuevo material. Se constituye en una síntesis de los puntos clave del informe, y se establecen posibles direcciones futuras.

\section{Conclusión}
El autor interpreta los resultados más relevantes obtenidos.

\section{Notas}
Esta plantilla fue realizada en base al recuso publicado por CyTA titulado: Informes científicos y técnicos: detalles para su elaboración \cite{Perissé2019}; y a la revisión de literatura y desarrollos técnicos de ChatGPT\cite{ChatGPT2025}

.


\printbibliography % Imprime las referencias automáticamente desde el archivo .bib


\end{document}


.


\printbibliography % Imprime las referencias automáticamente desde el archivo .bib


\end{document}
